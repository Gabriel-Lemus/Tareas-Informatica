\documentclass[11pt,letterpaper]{article}

\usepackage[utf8]{inputenc}
\usepackage[spanish]{babel}
\usepackage{amsmath}
\usepackage{amsfonts}
\usepackage{amssymb}
\usepackage[left=3cm,right=3cm,top=2cm,bottom=3cm]{geometry}

\linespread{1.1}

\begin{document}

\title{Hoja de Trabajo \#1 - Informática 1}
\author{Jeremy Cáceres y Gabriel Lemus}
\date{}
\maketitle

\section*{\textbf{Ejercicio \#2 (Abstracción)}}
\textbf{Dado:}

\begin{itemize}
\item Figura geométrica de 6 caras planas.
\item Cada cara es adyacente a 4 caras diferentes.
\item Las sumas de los  lados opuestos opuestas del dado es 7.
\item Solamente puede girarse un dado a la vez.
\item Cada movimiento sólo puede ser de 90 grados a lo largo del eje transversal o anteroposterior.
\item El movimiento no puede realizarse en el eje longitudinal.
\end{itemize}

\subsection*{\textbf{\Large Modelo de los dados y sus transiciones como un \emph{grafo}.}}
\subsubsection*{\textbf{1. \emph{Conjunto} de nodos del \emph{grafo:}}}

(Posibles posiciones en la que pueden estar los dados en un momento dado).
\bigskip

\ \ \ \ \ \ \ \ \ \ \ \ \ \ \ \ \ \ \ \ (1,1)\ \ \ \ \ (2,2)\ \ \ \ \ (3,3)\ \ \ \ \ (4,4)\ \ \ \ \ (5,5)\ \ \ \ \ (6,6)

\ \ \ \ \ \ \ \ \ \ \ \ \ \ \ \ \ \ \ \ (1,2)\ \ \ \ \ (2,3)\ \ \ \ \ (3,4)\ \ \ \ \ (4,5)\ \ \ \ \ (5,6)

\ \ \ \ \ \ \ \ \ \ \ \ \ \ \ \ \ \ \ \ (1,3)\ \ \ \ \ (2,4)\ \ \ \ \ (3,5)\ \ \ \ \ (4,6)

\ \ \ \ \ \ \ \ \ \ \ \ \ \ \ \ \ \ \ \ (1,4)\ \ \ \ \ (2,5)\ \ \ \ \ (3,6)

\ \ \ \ \ \ \ \ \ \ \ \ \ \ \ \ \ \ \ \ (1,5)\ \ \ \ \ (2,6)

\ \ \ \ \ \ \ \ \ \ \ \ \ \ \ \ \ \ \ \ (1,6)

\subsubsection*{\textbf{2. \emph{Conjunto} de vértices del \emph{grafo:}}}

(Posibles transiciones de los dados, moviendo sólo un dado a la vez 90 grados).
\bigskip

\noindent$<(1,1),(1,3)>,\ \ \ <(2,2),(2,4)>,\ \ \ <(3,3),(3,6)>,\ \ \ <(5,4),(5,5)>,\ \ \ <(6,6),(4,6)>,$

\noindent$<(1,3),(1,5)>,\ \ \ <(2,4),(2,6)>,\ \ \ <(3,6),(3,5)>,\ \ \ <(5,5),(5,6)>,\ \ \ <(4,6),(4,4)>$

\noindent$<(1,5),(1,6)>,\ \ \ <(2,6),(2,5)>,\ \ \ <(3,5),(3,4)>,\ \ \ <(5,6),(6,6)>,$

\noindent$<(1,6),(1,4)>,\ \ \ <(2,5),(2,4)>,\ \ \ <(3,4),(5,4)>,$

\noindent$<(1,4),(1,2)>,\ \ \ <(2,3),(3,3)>,$

\noindent$<(1,2),(2,2)>,$

\pagebreak

\section*{\textbf{Ejercicio \#3}}
\textbf{• ¿Qué estructura de datos podría representar un lanzamiento de dados?}

\noindent Una lista de conjuntos de situaciones que representan las posibles combinaciones que pueden tomar ambos dados, sin repetir combinaciones y donde se sólo se puede girar un dado a la vez, 90 grados sobre cualquiera de sus ejes, excepto en el eje longitudinal. \bigskip

\textbf{• ¿Qué algoritmo se puede utilizar para generar dicha estructura?}

\begin{enumerate}
\item Empezar con los dados de forma que sus posiciones sean (1,1).
\item Girar cualquiera de los dos dados 90 grados en cualquier de las cuatro direcciones posibles ($\uparrow, \ \downarrow, \ \rightarrow, $ o $ \leftarrow$). Estas direcciones están dadas por los ejes transversal y anteroposterior.
\item Girar el mismo dado 90 grados en una dirección perpendicular al movimiento anterior. (Por ejemplo, si el primer movimiento fue $\uparrow$, el segundo movimiento puede ser $\rightarrow$ o $\leftarrow$).
\item Repetir el movimiento realizado en el paso 2, seguido del 3, finalizando con el 2. (Por ejemplo, si el primer movimiento fue $\uparrow$ y el segundo movimiento fue $\leftarrow$; los movimientos totales a realizar en el dado elegido, serían: $\uparrow, \ \leftarrow, \ \uparrow, \ \leftarrow, \ \uparrow$).
\item Mueva el otro dado a un número en el que no había estado anteriormente.
\item Realice los siguientes pasos con el otro dado: invierta la dirección de los movimientos realizados en el paso 4 y omita el último movimiento. (En el ejemplo presentado, los movimientos del paso 4 fueron, $\uparrow, \ \leftarrow, \ \uparrow, \ \leftarrow, \ \uparrow$. Por lo tanto los siguientes movimientos serían: $\downarrow, \ \rightarrow, \ \downarrow, \ \rightarrow$).
\item Mueva el otro dado a un número en el que no había estado anteriormente.
\item Continúe con los pasos 6 y 7 hasta alcanzar todas las posibles combinaciones.
\end{enumerate}

\textbf{• ¿Como es posible asegurarse que el algoritmo siempre produce un resultado?}

\noindent Porque el algoritmo limita los movimientos que se pueden realizar, de forma busca que no se repitan combinaciones ya obtenidas. (Es por ello que en el paso 6 se omite el último movimiento, percisamente para impedir que esto suceda).

\end{document}